
\documentclass[a4paper,man,british]{apa6}
\usepackage[british]{babel}
\usepackage[utf8]{inputenc}
\usepackage{csquotes}
\usepackage[hidelinks]{hyperref}
\usepackage[style=apa]{biblatex}
\DeclareLanguageMapping{british}{british-apa}

% maps apacite commands to biblatex commands
\let \citeNP \cite
\let \citeA \textcite
\let \cite \parencite

\addbibresource{zotero_references.bib}
\addbibresource{bibliography.bib}

\title{Dehumanization of patients: Are patients people ?}
\shorttitle{Dehumanization of patients}
\author{Austin Paul}
\affiliation{RMIT UNIVERSITY \\ s3634517 \\ Word Count : 986}




\begin{document}

\maketitle

\section{Introduction} %200

This paper attempts explore the phenomenon of dehumanisation of patients in medical and care recipient settings. Rather than attributing blame to individuals this paper tries to 

\section{Definition of identified issue} %400


% reflection
On a tour of Port Arthur historic sites, the tour guide showed us a a fairly intact building where the colonial government kept their worst of the worst criminal's, a place where convicts were stripped of their identity as an individual and imposed a number, by which he will be hence addressed by. A punishment so atrocious to the human nature that it inspired callous hearts to repent and transcend the pain of whip lashes. what this reminded me of was how we as nurses may be guilty of similar lapses when we talk about our patients as room numbers or when some people sometimes somehow seem to not see patients as people but as a personified object. 


\newpage
\section{Overview of literature supporting issue} %600

Patients are not fully human: a nurse's coping response to stress

Defensive dehumanization in the medical practice: A cross‐sectional study from a health care worker's perspective

‘All the services were excellent. It is when the human element comes in that things go wrong’: dissatisfaction with hospital care in the last year of life

Providing a “Good Death”: Critical Care Nurses’ Suggestions for Improving End-of-Life Care

In their own words: Patients and families define high-quality palliative care in the intensive care unit*

Power increases dehumanization

\newpage
\section{Discussion} %600


% the care givers role fundametaly is to know better than the patien

\newpage
\section{Conclusion}%200



%\printbibliography

\end{document}
