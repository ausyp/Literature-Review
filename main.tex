
\documentclass[a4paper,man,british]{apa6}
\usepackage[british]{babel}
\usepackage[utf8]{inputenc}
\usepackage{csquotes}
\usepackage[hidelinks]{hyperref}
\usepackage[style=apa]{biblatex}
\DeclareLanguageMapping{british}{british-apa}

% maps apacite commands to biblatex commands
\let \citeNP \cite
\let \citeA \textcite
\let \cite \parencite


\addbibresource{zotero_references.bib}
\addbibresource{bibliography.bib}

\title{Dehumanization of patients: Are patients people ?}
\shorttitle{Dehumanization of patients}
\author{Austin Paul}
\affiliation{RMIT UNIVERSITY \\ s3634517 \\ Word Count : 1900}




\begin{document}

\maketitle

\section{Introduction} %200
Dehumanisation is generally viewed as a negative phenomenon and in the medical setting it is observed in many aspects concerning the dignity of the patient. 
This paper attempts to explore the phenomenon of dehumanisation of patients in medical and care recipient settings. Dehumanisation of patients happens when patients are not viewed  as fully humans by a health professional. Rather than attributing blame to individual stakeholders this paper tries to explore the institutional basis for this phenomenon and to examine how organisational structures aid in dehumanisation.  A journal article by \textcite{haque_dehumanization_2012} is used as the key literature for this literature appraisal. The reviews are based on the definitions from this article.

\section{Definition of the identified issue} %400

\textcite{haslam_recent_2016} define dehumanisation as the act of perceiving or treating people as if they are less that fully human. \textcite{lammers_power_2011} defined it as the  act of denying humans their human nature and treating them like objects. \textcite{haque_dehumanization_2012} state that the essence of dehumanisation is the denial of distinctively human mind to another person they, further emphasise that the process of dehumanisation denies the victims mind the dimensions of \textit{experience} or \textit{agency} and sometime both. \textit{Experience} being the capacity to feel pleasure and pain, and \textit{agency} being the ability to plan, intend and exert choice.
The issue that has been identified here is the dehumanisation of patients in medical and care receiving setting.The core of the question being asked is would these individuals (patients) be treated the same way had they not been a patients at the hospital.
% reflection
\\
On a tour of Port Arthur historic sites, the tour guide showed us a a fairly intact building where the colonial government kept their worst of the worst criminal's, a place where convicts were stripped of their identity as an individual and imposed a number, by which he will be hence addressed by. A punishment so atrocious to the human nature that it inspired callous hearts to repent and transcend the pain of whip lashes. What this reminded me of was how we as nurses may be guilty of similar lapses when we talk about our patients as room numbers or when some people sometimes somehow seem to not see patients as people but as a personified object. 
\\
 The search strategy used to identify relevant literature for this paper involved searching the Scopus electronic journal database and google scholar database. Keywords such as "dehumanisation of patient","patients","dehumanisation","dehumanization" where used. The results where primarily sorted by relevance and  the number of citations. Primary research were then identified and six relevant from that were selected based on research question and setting. The selected papers were ran against a retraction watch database to see if there where any retractions.

\section{Overview of the literature supporting the issue} %600

%Patients are not fully human: a nurse's coping response to stress
\textcite{trifiletti_patients_2014} Conducted a questionnaire based anonymous survey with the voluntary participation of one hundred and nine nurses working in a central Italian town hospital.
The researchers delivered 220 survey packages which included the questionnaire, A letter explaining the aims of research guaranteeing anonymity. 109 of the 220 questionnaires were returned and were used for this study. The questions asked were mainly focused to determine the co-relation between dehumanisation of patients and stress experienced by staff. The researchers used multiple variables such as humaneness attribution, commitment to organisation, commitment to patient and stress. The research found that there is a strong co-relation between a nurses commitment to patients and the nurses stress levels and conversely nurses who showed higher commitment to organisation had lower stress levels, they also found that more the nurse attributes uniquely human characteristics to a patient higher the stress levels. From this observation the researchers hypothesised that viewing patients as less than human makes their sufferings more bearable and protects nurses from stress. 
A major limitation of this study is the statistical principle that co-relation does not imply causation \cite{altman_association_2015}. Due to the co-relational nature of the study the research were unable to come to a conclusion further, the study was limited to just one hospital and therefore the setting prevent generalisation to wider population.
\\
%Defensi unave dble ehumers anizwhereation in the medical practice: A cross‐sectional study from a health care worker's perspective
Conversely, \textcite{vaes_defensive_2013} showed in their study that, subtle dehumanisation patients has some benefits in some instances. They conducted a quantitative methodology based study in four Italian health care institution with the voluntary participation of professional staff (N=78) of this (N=27) were professional nurses. In the study the participants where asked to fill-out a questionnaire which had a fictitious story of a patient with end stage cancer and a list of 28 uniquely human and non-uniquely human emotions. The participants where requested to tick the emotions they experienced. This was then used to determine how they attributed humaneness the questionnaire also had  questions to determine the professionals extend of patient contact and burnout they experienced. The limitations of this study are the co-relational nature of the data obtained and the consequent inability to assert a firm conclusion. It is also worth noting that professionals who institutionally have less contact with patients like sanitation workers were also included in the survey and may have diluted the results. What the study did show is that there is a strong co-relation between staff burnout and attribution of humanness to patients and that subtle dehumanisation can be functional for professionals where suffering and pain are recurrent reality.
 \\
%Power increases dehumanisation
\textcite{lammers_power_2011} conducted qualitative research exploring the relationship between power and dehumanisation. The research consisted of three study's with the participation of  students from a university in the Netherlands. The third study in this research has significant relevance to issue at hand, as it examine the relationship between power and dehumanisation in medical setting. This study (n=50) university students were instructed that they would participate in study on medical decision making and were asked about best way to treat a patient after being presented with patient files. Participants in a high power condition were asked to play the role of senior surgeon and Junior surgeon and participants in low power condition were asked to play the role of nurses. The study found that there is a strong association between power and dehumanisation. The participants in high power positions consistently made tougher treatment decisions for the patient and justified it by dehumanising the patients. The high power participants disregard for the patients emotions while supporting tougher decisions exemplifies that the imbalance of power between care providers and patients also contribute in dehumanisation of patients. The major limitation for this study is its setting and the fact that the participants were students not professionals.
\\
%‘All the services were excellent. It is when the human element comes in that things go wrong’: dissatisfaction with hospital care in the last year of life
%
%In their own words: Patients and families define high-quality palliative care in the intensive care unit*
%Nurses’ and patients’ perceptions of dignity
\textcite{walsh_nurses_2002} conducted unstructured interviews with four nurses and five patients from metropolitan hospital in South Australia to explore the nurses and patients perception of dignity. The participants were recruited by placing notices in the hospital calling for volunteers. The interviews were audio taped and then transcribed by the researchers and the emerging themes from these transcriptions were then further analysed. The questions where designed to explore participants (both nurses and patients) experiences where they feel the patients dignity was maintained or compromised. Surprisingly the study has shown that nurses and patients have shared perceptions of dignity. Both the patients and nurses complained about the lack of time and about being rushed. Some nurses expressed that the overall hospital system is not supportive of patient dignity from an institutional perspective and that they not seen as people. The patients accounts also show that they felt that they were not seen as a person in some circumstances. This research brings the much needed perspective of patients into the discussion and shows dehumanisation is real and mutually experienced. The major limitation of this study is its small sample size but the themes expressed are generalisable to wider population.
\\
%Providing a “Good Death”: Critical Care Nurses’ Suggestions for Improving End-of-Life Care
 
 \textcite{beckstrand_providing_2006} conducted a qualitative survey in the United states in which a random sample of (n=1409) members of the American Association of Critical-care nurses where sent a survey questionnaire with an open ended question regarding how they would change end-of life care in the ICU. 861 nurses returned the questionnaire and 530 had some kind responses on the questionnaire and of this 530, (n=485) respondents had offered one or more suggestions in their response. These responses were then coded and synthesised by two researchers who had a mean critical care experience of 19 years. Due to the random sampling, there is no proof of the assumption that all the respondents are nurses. Another limitation of the study is that this is limited to just the staffs perspective. Even though in the surface this study seems to be talking about improving end-of life care for palliative patients, the barriers experienced by staff to provide a humane care are institutional limitations like staffing patterns and lack of time or short staffing. These institutional limitations are factors that contribute to the dehumanisation of patients. %core literature ref here.


\section{Discussion} %600

The over arching theme of the literature overview is that the dehumanisation of patients is not a result of malicious intent from care providers but a combination of institutional factors and conventional precedents as identified by \citeauthor{haque_dehumanization_2012}. \citeauthor{trifiletti_patients_2014} Showed that nurses dehumanise patients as a coping mechanism towards stress caused by patients suffering and work pressure. \citeauthor{vaes_defensive_2013} found that nurse burnout and attribution of humane characteristics to the patients have a strong association. \citeauthor{walsh_nurses_2002} showed that both the patients and nurses are intrinsically aware of the unintentional dehumanisation and the institutional limitations. \citeauthor{beckstrand_providing_2006} demonstrated that an overwhelming number of respondents to their survey identified institutional barriers as detrimental to the care they wanted to provide. It is as though the present institutional structures does not afford nurses and care givers an opportunity to provide the care they want to provide.
\citeauthor{lammers_power_2011} demonstrated a proportional relation between power and dehumanisation. The imbalance of power between care providers and care recipient's put the recipient in a vulnerable position especially when those professional's may have reduced empathy as shown by \textcite{cheng2007expertise}.
\\
The biggest change that is required is unsurprisingly, more staffing allowing nurses more time to provide individualised care. Organisation should also make support available for nurses experiencing burnout and prevent them from adopting harmfull coping mechanisms.
\\
There is also another change that looks more and more self evident. The nursing and medical institution as we know it today has some of its origin in the military \cite{egenes2017history}. A lot of nursing practices are the legacy of military nursing. Tools like ISBAR that are widely used now where co-opted from the military \cite{johnson2012exploring}. With the numerous advantages that came with the military legacy, there where also the rigid hierarchical structures that are present at hospitals. These structures contribute to the imbalance of power that aids dehumanisation. There is a need to further civillianize the medical institution's and reduce the disparity between patients and care providers as suggested by \citeauthor{haque_dehumanization_2012}. And further strive for moral engagement by referring to patients by people.
\\

% the over arching theme is that a lot dehumanization happens as a coping mechanism for reducing emotional burnout
% the care givers role fundametaly is to know better than the patien
% Law like eu gdpr all information regarding the patient that a hospital or institusion has should be made available to the patient.

\section{Conclusion}%200

Dehumanisation of patients is a very complicated and multi-dimensional phenomenon. The factors that contribute to it are also being identified, further as the understanding of this phenomenon expands. This paper attempted to focus on the institutional factors and the underlying imbalance of power. It is apparent that dehumanisation of patients is a common occurrence and needs to be addressed. Further research is needed to integrate both the nurses and patients experiences and draw more conclusive results. It may also be interesting to have some research into the commercial viability of the changes being proposed.
\printbibliography

\end{document}
