
\documentclass[a4paper,man,british]{apa6}
\usepackage[british]{babel}
\usepackage[utf8]{inputenc}
\usepackage{csquotes}
\usepackage[hidelinks]{hyperref}
\usepackage[style=apa]{biblatex}
\DeclareLanguageMapping{british}{british-apa}

% maps apacite commands to biblatex commands
\let \citeNP \cite
\let \citeA \textcite
\let \cite \parencite

\addbibresource{zotero_references.bib}
\addbibresource{bibliography.bib}

\title{Dehumanization of patients: Are patients people ?}
\shorttitle{Dehumanization of patients}
\author{Austin Paul}
\affiliation{RMIT UNIVERSITY \\ s3634517 \\ Word Count : 986}




\begin{document}

\maketitle

\section{Introduction} %200

This paper attempts explore the phenomenon of dehumanisation of patients in medical and care recipient settings. Rather than attributing blame to individuals this paper tries to 

\section{Definition of identified issue} %400

\textcite{haslam_recent_2016} defines dehumanisation as the act of perceiving or treating people as if they are less that fully human. \textcite{lammers_power_2011} defined it as the  act of denying humans their human nature and treating them like objects. \textcite{haque_dehumanization_2012} states that the essence of dehumanisation is the denial of distinctively human mind to another person they, further emphasise that the process of dehumanisation denies the victims mind the dimensions of \textit{experience} or \textit{agency} and sometime both. \textit{Experience} being the capacity to feel pleasure and pain, and \textit{agency} being the ability to plan, intend and exert choice.
The issue that has been identified here is the dehumanisation of patients in medical and care receiving setting.The core of the question being asked is would these individuals (patients) be treated the same way had they not been a patients at the hospital.
% reflection
\\
On a tour of Port Arthur historic sites, the tour guide showed us a a fairly intact building where the colonial government kept their worst of the worst criminal's, a place where convicts were stripped of their identity as an individual and imposed a number, by which he will be hence addressed by. A punishment so atrocious to the human nature that it inspired callous hearts to repent and transcend the pain of whip lashes. what this reminded me of was how we as nurses may be guilty of similar lapses when we talk about our patients as room numbers or when some people sometimes somehow seem to not see patients as people but as a personified object. 
\\
 The search strategy used to identify relevant literature for this paper involved searching the Scopus electronic journal database and google scholar database. Keywords such as "dehumanisation of patient","patients","dehumanisation","dehumanization" where used. The results where primarily sorted by relevance and  the number of citations. Primary research were then identified and six relevant from that were selected based on research question and setting. The selected papers were ran against a retraction watch database to see if there where any retractions.
\newpage
\section{Overview of literature supporting issue} %600

%Patients are not fully human: a nurse's coping response to stress
\textcite{trifiletti_patients_2014} Conducted a questionnaire based anonymous survey with the voluntary participation of one hundred and nine nurses working in a central Italian town hospital.
The researchers delivered 220 survey packages which included the questionnaire, A letter explaining the aims of research guaranteeing anonymity. 109 of the 220 questionnaires were returned and were used for this study. The questions asked were mainly focused to determine the co-relation between dehumanisation of patients and stress experienced by staff. The researchers used multiple variables such as humaneness attribution, commitment to organisation, commitment to patient and stress. The research found that there is a strong co-relation between a nurses commitment to patients and the nurses stress levels and conversely nurses who showed higher commitment to organisation had lower stress levels, they also found that more the nurse attributes uniquely human characteristics to a patient higher the stress levels. From this observation the researchers hypothesised that viewing patients as less than human makes their sufferings more bearable and protects nurses from stress. 
A major limitation of this study is the statistical principle that co-relation does not imply causation \cite{altman_association_2015}. Due to the co-relational nature of the study the research were unable to come to a conclusion further, the study was limited to just one hospital and therefore the setting prevent generalisation to wider population.
\\
%Defensi unave dble ehumers anizwhereation in the medical practice: A cross‐sectional study from a health care worker's perspective
Conversely, \textcite{vaes_defensive_2013} showed in their study that, subtle dehumanisation patients has some benefits in some instances. They conducted a quantitative methodology based study in four Italian health care institution with the voluntary participation of professional staff (N=78) of this (N=27) were professional nurses. In the study the participants where asked to fill-out a questionnaire which had a fictitious story of a patient with end stage cancer and a list of 28 uniquely human and non-uniquely human emotions. The participants where requested to tick the emotions they experienced. This was then used to determine how they attributed humaneness the questionnaire also had  questions to determine the professionals extend of patient contact and burnout they experienced. The limitations of this study are the co-relational nature of the data obtained and the consequent inability to assert a firm conclusion. It is also worth noting that professionals who institutionally have less contact with patients like sanitation workers were also included in the survey and may have diluted the results. What the study did show is that there is a strong co-relation between staff burnout and attribution of humanness to patients and that subtle dehumanisation can be functional for professionals where suffering and pain are recurrent reality.
\\
%Providing a “Good Death”: Critical Care Nurses’ Suggestions for Improving End-of-Life Care
 
 \textcite{beckstrand_providing_2006} conducted another qualitative survey in the United states in which a random sample of (n=1409) members of the American Association of Critical-care nurses where sent a survey questionnaire with an open ended question regarding how they would change end-of life care in the ICU. 861 nurses returned the questionnaire and 530 had some kind responses on the questionnaire and of this 530, (n=485) respondents had offered 1 or more suggestions in their response. These responses were then coded and synthesised by two researchers who had a mean critical care experience of 19 years. Due to the random sampling, there is no proof of the assumption that all the respondents are nurses. Another limitation of the study is that this is limited to just the staffs perspective. Even though in the surface this study seems to be talking about improving end-of life care for palliative patients, the barriers experienced by staff to provide a humane care are institutional limitations like staffing patterns and lack of time or short staffing. These institutional limitations are factors that contribute to the dehumanisation of patients. %core literature ref here.
 \\
%Power increases dehumanisation
\textcite{lammers_power_2011}

%‘All the services were excellent. It is when the human element comes in that things go wrong’: dissatisfaction with hospital care in the last year of life

%In their own words: Patients and families define high-quality palliative care in the intensive care unit*



\newpage
\section{Discussion} %600

% the over arching theme is that a lot dehumanization happens as a coping mechanism for reducing emotional burnout
% the care givers role fundametaly is to know better than the patien
% Law like eu gdpr all information regarding the patient that a hospital or institusion has should be made available to the patient.
\newpage
\section{Conclusion}%200



%\printbibliography

\end{document}
